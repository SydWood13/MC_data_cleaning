% Options for packages loaded elsewhere
\PassOptionsToPackage{unicode}{hyperref}
\PassOptionsToPackage{hyphens}{url}
%
\documentclass[
]{article}
\usepackage{amsmath,amssymb}
\usepackage{iftex}
\ifPDFTeX
  \usepackage[T1]{fontenc}
  \usepackage[utf8]{inputenc}
  \usepackage{textcomp} % provide euro and other symbols
\else % if luatex or xetex
  \usepackage{unicode-math} % this also loads fontspec
  \defaultfontfeatures{Scale=MatchLowercase}
  \defaultfontfeatures[\rmfamily]{Ligatures=TeX,Scale=1}
\fi
\usepackage{lmodern}
\ifPDFTeX\else
  % xetex/luatex font selection
\fi
% Use upquote if available, for straight quotes in verbatim environments
\IfFileExists{upquote.sty}{\usepackage{upquote}}{}
\IfFileExists{microtype.sty}{% use microtype if available
  \usepackage[]{microtype}
  \UseMicrotypeSet[protrusion]{basicmath} % disable protrusion for tt fonts
}{}
\makeatletter
\@ifundefined{KOMAClassName}{% if non-KOMA class
  \IfFileExists{parskip.sty}{%
    \usepackage{parskip}
  }{% else
    \setlength{\parindent}{0pt}
    \setlength{\parskip}{6pt plus 2pt minus 1pt}}
}{% if KOMA class
  \KOMAoptions{parskip=half}}
\makeatother
\usepackage{xcolor}
\usepackage[margin=1in]{geometry}
\usepackage{color}
\usepackage{fancyvrb}
\newcommand{\VerbBar}{|}
\newcommand{\VERB}{\Verb[commandchars=\\\{\}]}
\DefineVerbatimEnvironment{Highlighting}{Verbatim}{commandchars=\\\{\}}
% Add ',fontsize=\small' for more characters per line
\usepackage{framed}
\definecolor{shadecolor}{RGB}{248,248,248}
\newenvironment{Shaded}{\begin{snugshade}}{\end{snugshade}}
\newcommand{\AlertTok}[1]{\textcolor[rgb]{0.94,0.16,0.16}{#1}}
\newcommand{\AnnotationTok}[1]{\textcolor[rgb]{0.56,0.35,0.01}{\textbf{\textit{#1}}}}
\newcommand{\AttributeTok}[1]{\textcolor[rgb]{0.13,0.29,0.53}{#1}}
\newcommand{\BaseNTok}[1]{\textcolor[rgb]{0.00,0.00,0.81}{#1}}
\newcommand{\BuiltInTok}[1]{#1}
\newcommand{\CharTok}[1]{\textcolor[rgb]{0.31,0.60,0.02}{#1}}
\newcommand{\CommentTok}[1]{\textcolor[rgb]{0.56,0.35,0.01}{\textit{#1}}}
\newcommand{\CommentVarTok}[1]{\textcolor[rgb]{0.56,0.35,0.01}{\textbf{\textit{#1}}}}
\newcommand{\ConstantTok}[1]{\textcolor[rgb]{0.56,0.35,0.01}{#1}}
\newcommand{\ControlFlowTok}[1]{\textcolor[rgb]{0.13,0.29,0.53}{\textbf{#1}}}
\newcommand{\DataTypeTok}[1]{\textcolor[rgb]{0.13,0.29,0.53}{#1}}
\newcommand{\DecValTok}[1]{\textcolor[rgb]{0.00,0.00,0.81}{#1}}
\newcommand{\DocumentationTok}[1]{\textcolor[rgb]{0.56,0.35,0.01}{\textbf{\textit{#1}}}}
\newcommand{\ErrorTok}[1]{\textcolor[rgb]{0.64,0.00,0.00}{\textbf{#1}}}
\newcommand{\ExtensionTok}[1]{#1}
\newcommand{\FloatTok}[1]{\textcolor[rgb]{0.00,0.00,0.81}{#1}}
\newcommand{\FunctionTok}[1]{\textcolor[rgb]{0.13,0.29,0.53}{\textbf{#1}}}
\newcommand{\ImportTok}[1]{#1}
\newcommand{\InformationTok}[1]{\textcolor[rgb]{0.56,0.35,0.01}{\textbf{\textit{#1}}}}
\newcommand{\KeywordTok}[1]{\textcolor[rgb]{0.13,0.29,0.53}{\textbf{#1}}}
\newcommand{\NormalTok}[1]{#1}
\newcommand{\OperatorTok}[1]{\textcolor[rgb]{0.81,0.36,0.00}{\textbf{#1}}}
\newcommand{\OtherTok}[1]{\textcolor[rgb]{0.56,0.35,0.01}{#1}}
\newcommand{\PreprocessorTok}[1]{\textcolor[rgb]{0.56,0.35,0.01}{\textit{#1}}}
\newcommand{\RegionMarkerTok}[1]{#1}
\newcommand{\SpecialCharTok}[1]{\textcolor[rgb]{0.81,0.36,0.00}{\textbf{#1}}}
\newcommand{\SpecialStringTok}[1]{\textcolor[rgb]{0.31,0.60,0.02}{#1}}
\newcommand{\StringTok}[1]{\textcolor[rgb]{0.31,0.60,0.02}{#1}}
\newcommand{\VariableTok}[1]{\textcolor[rgb]{0.00,0.00,0.00}{#1}}
\newcommand{\VerbatimStringTok}[1]{\textcolor[rgb]{0.31,0.60,0.02}{#1}}
\newcommand{\WarningTok}[1]{\textcolor[rgb]{0.56,0.35,0.01}{\textbf{\textit{#1}}}}
\usepackage{graphicx}
\makeatletter
\def\maxwidth{\ifdim\Gin@nat@width>\linewidth\linewidth\else\Gin@nat@width\fi}
\def\maxheight{\ifdim\Gin@nat@height>\textheight\textheight\else\Gin@nat@height\fi}
\makeatother
% Scale images if necessary, so that they will not overflow the page
% margins by default, and it is still possible to overwrite the defaults
% using explicit options in \includegraphics[width, height, ...]{}
\setkeys{Gin}{width=\maxwidth,height=\maxheight,keepaspectratio}
% Set default figure placement to htbp
\makeatletter
\def\fps@figure{htbp}
\makeatother
\setlength{\emergencystretch}{3em} % prevent overfull lines
\providecommand{\tightlist}{%
  \setlength{\itemsep}{0pt}\setlength{\parskip}{0pt}}
\setcounter{secnumdepth}{-\maxdimen} % remove section numbering
\ifLuaTeX
  \usepackage{selnolig}  % disable illegal ligatures
\fi
\usepackage{bookmark}
\IfFileExists{xurl.sty}{\usepackage{xurl}}{} % add URL line breaks if available
\urlstyle{same}
\hypersetup{
  pdftitle={NPB Collaboration Analysis Plan},
  pdfauthor={Sydney Wood},
  hidelinks,
  pdfcreator={LaTeX via pandoc}}

\title{NPB Collaboration Analysis Plan}
\author{Sydney Wood}
\date{2024-09-17}

\begin{document}
\maketitle

\section{Descriptives}\label{descriptives}

Summarize the characteristics of the sample. Especially first generation
status, native language, english proficiency, SES and racial
composition.

\section{Confirmatory analysis Multilevel
Model}\label{confirmatory-analysis-multilevel-model}

\subsection{Creating fake data for
analysis}\label{creating-fake-data-for-analysis}

\begin{Shaded}
\begin{Highlighting}[]
\NormalTok{packages }\OtherTok{\textless{}{-}} \FunctionTok{c}\NormalTok{(}\StringTok{"ggplot2"}\NormalTok{,  }\StringTok{"tidyverse"}\NormalTok{, }\StringTok{"apaTables"}\NormalTok{, }\StringTok{"colourpicker"}\NormalTok{, }\StringTok{"dplyr"}\NormalTok{, }\StringTok{"gridExtra"}\NormalTok{, }\StringTok{"knitr"}\NormalTok{, }\StringTok{"lme4"}\NormalTok{, }\StringTok{"reshape2"}\NormalTok{, }\StringTok{"stargazer"}\NormalTok{, }\StringTok{"gtsummary"}\NormalTok{, }\StringTok{"performance"}\NormalTok{, }\StringTok{"afex"}\NormalTok{, }\StringTok{"jtools"}\NormalTok{, }\StringTok{"svglite"}\NormalTok{)}

\CommentTok{\# Install packages not yet installed}
\NormalTok{installed\_packages }\OtherTok{\textless{}{-}}\NormalTok{ packages }\SpecialCharTok{\%in\%} \FunctionTok{rownames}\NormalTok{(}\FunctionTok{installed.packages}\NormalTok{())}
\ControlFlowTok{if}\NormalTok{ (}\FunctionTok{any}\NormalTok{(installed\_packages }\SpecialCharTok{==} \ConstantTok{FALSE}\NormalTok{)) \{}
  \FunctionTok{install.packages}\NormalTok{(packages[}\SpecialCharTok{!}\NormalTok{installed\_packages])}
\NormalTok{\}}

\CommentTok{\# Packages loading}
\FunctionTok{invisible}\NormalTok{(}\FunctionTok{lapply}\NormalTok{(packages, library, }\AttributeTok{character.only =} \ConstantTok{TRUE}\NormalTok{))}
\end{Highlighting}
\end{Shaded}

\begin{verbatim}
## Warning: package 'ggplot2' was built under R version 4.3.3
\end{verbatim}

\begin{verbatim}
## Warning: package 'tidyr' was built under R version 4.3.3
\end{verbatim}

\begin{verbatim}
## Warning: package 'readr' was built under R version 4.3.3
\end{verbatim}

\begin{verbatim}
## Warning: package 'dplyr' was built under R version 4.3.3
\end{verbatim}

\begin{verbatim}
## Warning: package 'stringr' was built under R version 4.3.3
\end{verbatim}

\begin{verbatim}
## -- Attaching core tidyverse packages ------------------------ tidyverse 2.0.0 --
## v dplyr     1.1.4     v readr     2.1.5
## v forcats   1.0.0     v stringr   1.5.1
## v lubridate 1.9.3     v tibble    3.2.1
## v purrr     1.0.2     v tidyr     1.3.1
## -- Conflicts ------------------------------------------ tidyverse_conflicts() --
## x dplyr::filter() masks stats::filter()
## x dplyr::lag()    masks stats::lag()
## i Use the conflicted package (<http://conflicted.r-lib.org/>) to force all conflicts to become errors
\end{verbatim}

\begin{verbatim}
## Warning: package 'colourpicker' was built under R version 4.3.2
\end{verbatim}

\begin{verbatim}
## 
## Attaching package: 'gridExtra'
## 
## The following object is masked from 'package:dplyr':
## 
##     combine
\end{verbatim}

\begin{verbatim}
## Warning: package 'knitr' was built under R version 4.3.3
\end{verbatim}

\begin{verbatim}
## Warning: package 'lme4' was built under R version 4.3.3
\end{verbatim}

\begin{verbatim}
## Loading required package: Matrix
## 
## Attaching package: 'Matrix'
## 
## The following objects are masked from 'package:tidyr':
## 
##     expand, pack, unpack
\end{verbatim}

\begin{verbatim}
## Warning in check_dep_version(): ABI version mismatch: 
## lme4 was built with Matrix ABI version 1
## Current Matrix ABI version is 0
## Please re-install lme4 from source or restore original 'Matrix' package
\end{verbatim}

\begin{verbatim}
## 
## Attaching package: 'reshape2'
## 
## The following object is masked from 'package:tidyr':
## 
##     smiths
## 
## 
## Please cite as: 
## 
##  Hlavac, Marek (2022). stargazer: Well-Formatted Regression and Summary Statistics Tables.
##  R package version 5.2.3. https://CRAN.R-project.org/package=stargazer
\end{verbatim}

\begin{verbatim}
## Warning: package 'gtsummary' was built under R version 4.3.3
\end{verbatim}

\begin{verbatim}
## Warning: package 'performance' was built under R version 4.3.3
\end{verbatim}

\begin{verbatim}
## Warning: package 'afex' was built under R version 4.3.3
\end{verbatim}

\begin{verbatim}
## ************
## Welcome to afex. For support visit: http://afex.singmann.science/
## - Functions for ANOVAs: aov_car(), aov_ez(), and aov_4()
## - Methods for calculating p-values with mixed(): 'S', 'KR', 'LRT', and 'PB'
## - 'afex_aov' and 'mixed' objects can be passed to emmeans() for follow-up tests
## - Get and set global package options with: afex_options()
## - Set sum-to-zero contrasts globally: set_sum_contrasts()
## - For example analyses see: browseVignettes("afex")
## ************
## 
## Attaching package: 'afex'
## 
## The following object is masked from 'package:lme4':
## 
##     lmer
\end{verbatim}

\begin{verbatim}
## Warning: package 'jtools' was built under R version 4.3.3
\end{verbatim}

\begin{verbatim}
## Warning: package 'svglite' was built under R version 4.3.3
\end{verbatim}

\begin{Shaded}
\begin{Highlighting}[]
\NormalTok{anon\_id }\OtherTok{\textless{}{-}} \FunctionTok{c}\NormalTok{(}\DecValTok{1}\SpecialCharTok{:}\DecValTok{300}\NormalTok{)}
\NormalTok{section }\OtherTok{\textless{}{-}} \FunctionTok{sample}\NormalTok{(}\DecValTok{1}\SpecialCharTok{:}\DecValTok{12}\NormalTok{, }\DecValTok{300}\NormalTok{, }\AttributeTok{replace =}\NormalTok{ T)}
\NormalTok{TA }\OtherTok{\textless{}{-}} \FunctionTok{sample}\NormalTok{(}\DecValTok{1}\SpecialCharTok{:}\DecValTok{3}\NormalTok{, }\DecValTok{300}\NormalTok{, }\AttributeTok{replace =}\NormalTok{ T)}
\NormalTok{Exam1 }\OtherTok{\textless{}{-}} \FunctionTok{rnorm}\NormalTok{(}\DecValTok{300}\NormalTok{, }\AttributeTok{mean =} \DecValTok{80}\NormalTok{, }\AttributeTok{sd =} \DecValTok{5}\NormalTok{)}
\NormalTok{Exam2 }\OtherTok{\textless{}{-}} \FunctionTok{rnorm}\NormalTok{(}\DecValTok{300}\NormalTok{, }\AttributeTok{mean =} \DecValTok{85}\NormalTok{, }\AttributeTok{sd =} \DecValTok{7}\NormalTok{)}

\NormalTok{fake\_data }\OtherTok{\textless{}{-}} \FunctionTok{data.frame}\NormalTok{(anon\_id, section, TA, Exam1, Exam2)}

\NormalTok{fake\_data }\OtherTok{\textless{}{-}}\NormalTok{ fake\_data }\SpecialCharTok{\%\textgreater{}\%} \FunctionTok{mutate}\NormalTok{(}\AttributeTok{TA =} \FunctionTok{case\_when}\NormalTok{(section }\SpecialCharTok{==} \DecValTok{1} \SpecialCharTok{\textasciitilde{}} \DecValTok{1}\NormalTok{, section }\SpecialCharTok{==} \DecValTok{2} \SpecialCharTok{\textasciitilde{}} \DecValTok{1}\NormalTok{, section }\SpecialCharTok{==} \DecValTok{3} \SpecialCharTok{\textasciitilde{}} \DecValTok{1}\NormalTok{, section }\SpecialCharTok{==} \DecValTok{4} \SpecialCharTok{\textasciitilde{}} \DecValTok{2}\NormalTok{, section }\SpecialCharTok{==} \DecValTok{5} \SpecialCharTok{\textasciitilde{}} \DecValTok{2}\NormalTok{, section }\SpecialCharTok{==} \DecValTok{6} \SpecialCharTok{\textasciitilde{}} \DecValTok{2}\NormalTok{, section }\SpecialCharTok{==} \DecValTok{7} \SpecialCharTok{\textasciitilde{}} \DecValTok{3}\NormalTok{, section }\SpecialCharTok{==} \DecValTok{8} \SpecialCharTok{\textasciitilde{}} \DecValTok{3}\NormalTok{, section }\SpecialCharTok{==} \DecValTok{9} \SpecialCharTok{\textasciitilde{}} \DecValTok{3}\NormalTok{, section }\SpecialCharTok{==} \DecValTok{10} \SpecialCharTok{\textasciitilde{}} \DecValTok{4}\NormalTok{, section }\SpecialCharTok{==} \DecValTok{11} \SpecialCharTok{\textasciitilde{}} \DecValTok{4}\NormalTok{, section }\SpecialCharTok{==} \DecValTok{12} \SpecialCharTok{\textasciitilde{}} \DecValTok{4}\NormalTok{))}

\NormalTok{fake\_data }\OtherTok{\textless{}{-}}\NormalTok{ fake\_data }\SpecialCharTok{\%\textgreater{}\%} \FunctionTok{mutate}\NormalTok{(}\AttributeTok{condition =} \FunctionTok{case\_when}\NormalTok{(section }\SpecialCharTok{==} \DecValTok{1} \SpecialCharTok{|}\NormalTok{ section }\SpecialCharTok{==} \DecValTok{5} \SpecialCharTok{|}\NormalTok{ section }\SpecialCharTok{==} \DecValTok{11} \SpecialCharTok{|}\NormalTok{ section }\SpecialCharTok{==} \DecValTok{8} \SpecialCharTok{\textasciitilde{}} \DecValTok{1}\NormalTok{, section }\SpecialCharTok{==} \DecValTok{3} \SpecialCharTok{|}\NormalTok{ section }\SpecialCharTok{==} \DecValTok{9} \SpecialCharTok{|}\NormalTok{ section }\SpecialCharTok{==} \DecValTok{4} \SpecialCharTok{|}\NormalTok{ section }\SpecialCharTok{==} \DecValTok{10} \SpecialCharTok{\textasciitilde{}} \DecValTok{2}\NormalTok{, section }\SpecialCharTok{==} \DecValTok{2} \SpecialCharTok{|}\NormalTok{ section }\SpecialCharTok{==} \DecValTok{6} \SpecialCharTok{|}\NormalTok{ section }\SpecialCharTok{==} \DecValTok{12} \SpecialCharTok{|}\NormalTok{ section }\SpecialCharTok{==} \DecValTok{7} \SpecialCharTok{\textasciitilde{}} \DecValTok{3}\NormalTok{))}

\NormalTok{fake\_data\_long }\OtherTok{\textless{}{-}} \FunctionTok{gather}\NormalTok{(fake\_data, }\AttributeTok{key =} \StringTok{"obs"}\NormalTok{,}\AttributeTok{value =} \StringTok{"score"}\NormalTok{ , Exam1, Exam2)}
\end{Highlighting}
\end{Shaded}

I didn't model any shared variance in the nested grouping structure so
the proposed models will report singularity. This is very unlikely to
occur in the real data, even if there are no inherent differences
between discussion groups. Therefore the code for fitting the model is
commented out

\subsection{Confirmatory Analysis}\label{confirmatory-analysis}

\begin{Shaded}
\begin{Highlighting}[]
\NormalTok{eq0 }\OtherTok{\textless{}{-}}\NormalTok{ (score }\SpecialCharTok{\textasciitilde{}}\NormalTok{ obs }\SpecialCharTok{+}\NormalTok{ (}\DecValTok{1}\SpecialCharTok{|}\NormalTok{anon\_id))}
\CommentTok{\#mod0 \textless{}{-} lmer(eq0, data = fake\_data\_long)}
\CommentTok{\#summary(mod0)}



\NormalTok{eq\_5 }\OtherTok{\textless{}{-}}\NormalTok{ (score }\SpecialCharTok{\textasciitilde{}}\NormalTok{ obs }\SpecialCharTok{+}\NormalTok{ condition }\SpecialCharTok{+}\NormalTok{ (}\DecValTok{1}\SpecialCharTok{|}\NormalTok{anon\_id))}
\CommentTok{\#mod\_5 \textless{}{-} lmer(eq\_5, data = fake\_data\_long)}
\CommentTok{\#summary(mod\_5)}


\NormalTok{eq1 }\OtherTok{\textless{}{-}}\NormalTok{ (score }\SpecialCharTok{\textasciitilde{}}\NormalTok{ obs}\SpecialCharTok{*}\NormalTok{condition }\SpecialCharTok{+}\NormalTok{ (}\DecValTok{1}\SpecialCharTok{|}\NormalTok{anon\_id))}
\CommentTok{\#mod1 \textless{}{-} lmer(eq1, data = fake\_data\_long)}
\CommentTok{\#summary(mod1)}


\NormalTok{eq2 }\OtherTok{\textless{}{-}}\NormalTok{ (score }\SpecialCharTok{\textasciitilde{}}\NormalTok{ obs}\SpecialCharTok{*}\NormalTok{condition }\SpecialCharTok{+}\NormalTok{ (}\DecValTok{1}\SpecialCharTok{|}\NormalTok{anon\_id) }\SpecialCharTok{+}\NormalTok{ (}\DecValTok{1}\SpecialCharTok{|}\NormalTok{anon\_id}\SpecialCharTok{:}\NormalTok{section))}
\CommentTok{\#mod2 \textless{}{-} lmer(eq2, data = fake\_data\_long)}
\CommentTok{\#summary(mod2)}

\NormalTok{eq3 }\OtherTok{\textless{}{-}}\NormalTok{ (score }\SpecialCharTok{\textasciitilde{}}\NormalTok{ obs}\SpecialCharTok{*}\NormalTok{condition }\SpecialCharTok{+}\NormalTok{ (}\DecValTok{1}\SpecialCharTok{|}\NormalTok{anon\_id) }\SpecialCharTok{+}\NormalTok{ (}\DecValTok{1}\SpecialCharTok{|}\NormalTok{anon\_id}\SpecialCharTok{:}\NormalTok{section) }\SpecialCharTok{+}\NormalTok{ (}\DecValTok{1}\SpecialCharTok{|}\NormalTok{anon\_id}\SpecialCharTok{:}\NormalTok{section}\SpecialCharTok{:}\NormalTok{TA))}
\CommentTok{\#mod3 \textless{}{-} lmer(eq3, data = fake\_data\_long)}
\CommentTok{\#summary(mod3)}
\end{Highlighting}
\end{Shaded}

\#Exploratory analyses

These models and analyses will be created post hoc as we do not have
specific hypotheses regarding effect of various demographic or course
related variables.

\end{document}
